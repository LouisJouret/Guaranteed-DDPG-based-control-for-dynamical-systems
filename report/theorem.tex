% Copyright (c) 2023 Louis Jouret
% 
% This software is released under the MIT License.
% https://opensource.org/licenses/MIT

\section{Theory}
\newframedtheorem{theorem}{Theorem}
\newframedtheorem{corollary}{Corollary}[theorem]
\begin{theorem}
Consider two closed sets $\mathcal{O}$ and $\mathcal{S}$, such that $\mathcal{O}$ is a practical set defined by
\begin{equation}
    \mathcal{O} = \left\{x : g_k(x) \leq 0, k = 1,2,\ldots,r \right\}
\end{equation}

Now consider the system $\dot{x} = f(x)$ and assume that for each initial
condition $x(0)$ in $\mathcal{S} \setminus \mathcal{O}$ it admits a unique solution defined for all $t \geq 0$. Then  
$\mathcal{S} \setminus \mathcal{O}$ is positively invariant for the system if and
only if the velocity vector satisfies:

\begin{equation}
    f(x) \notin \mathcal{T}_{\mathcal{O}}(x) \textit{, for all } x \in \partial\mathcal{O} \cap \mathcal{S}
    \label{eq:th1}
\end{equation}
\end{theorem}

Equation (\ref{eq:th1}) can be geometrically interpreted as
\begin{align*}
    f(x) \in \left\{z : \nabla g_{i}{(x)}^T z > 0 \text{, for all i} \in B(x)\right\} , \text{where } B(x) = \left\{i : g_i(x) = 0 \right\}
\end{align*}

\begin{corollary}
Under the assumption that the set $\mathcal{O}$ is a polytope defined by
\begin{equation}
    \mathcal{O} = \left\{x : C_{\mathcal{O}}x \leq d_{\mathcal{O}} \right\}
\end{equation}
Let's consider the system $\dot x = f(x)$. Then  
$\mathcal{S} \setminus \mathcal{O}$ is positively invariant for the 
system if and only if the velocity vector satisfies:

\begin{equation}
    f(x) \in \left\{z : \begin{bmatrix}
        \mathcal{C}_{\mathcal{O}_{i,0}}\\
        .\\
        .\\
        .\\
        \mathcal{C}_{\mathcal{O}_{i,n}}
    \end{bmatrix}^T z > 0 \text{, for all }i: \text{row}_i(\mathcal{C}_{\mathcal{O}}) \cdot x = d_{\mathcal{O}_i} \right\}
    \label{eq:col1}
\end{equation}
\end{corollary}
% Copyright (c) 2023 Louis Jouret
% 
% This software is released under the MIT License.
% https://opensource.org/licenses/MIT

\section{Theory}
\newframedtheorem{theorem}{Theorem}
\newframedtheorem{corollary}{Corollary}[theorem]

\begin{theorem}
Consider the closed set $\mathcal{O}$ and the system $\dot{x}(t) = f(x(t))$ and assume that for each initial
condition $x(0) \notin \mathcal{O}$ it admits a unique solution defined for all 
$t \geq 0$. Then $x(t) \notin \interior{\mathcal{O}}$ for $t \geq 0$ if and 
only if the velocity vector satisfies:

\begin{align*}
    f(x) \notin \interior{\mathcal{T}_{\mathcal{O}}(x)}, \text{for all } x \in \partial \mathcal{O}
\end{align*}
\end{theorem}

\begin{theorem}
Consider the pratical set $\mathcal{O}$ defined by
\begin{equation}
    \mathcal{O} = \left\{x : g_k(x) \leq 0, k = 1,2,\ldots,r \right\}
\end{equation}

Now consider the system $\dot{x}(t) = f(x(t))$ and assume that for each initial
condition $x(0) \notin \mathcal{O}$ it admits a unique solution defined for all 
$t \geq 0$. Then $x(t) \notin \interior{\mathcal{O}}$ for $t \geq 0$ if and 
only if the velocity vector satisfies:

\begin{align*}
    f(x) \in \left\{z : \nabla g_{i}{(x)}^T z \geq 0 \text{, for all i} \in B(x)\right\} , \text{where } B(x) = \left\{i : g_i(x) = 0 \right\}
\end{align*}
\end{theorem}


\begin{corollary}
If all the assumptions and conditions of Theorem 1 are verified and the set 
$\mathcal{O}$ is a polytope defined by
\begin{equation}
    \mathcal{O} = \left\{x : C_{\mathcal{O}}x \leq d_{\mathcal{O}} \right\}
\end{equation}
then $x(t) \notin \interior{\mathcal{O}}$ for $t \geq 0$ if and 
only if the velocity vector satisfies:

\begin{equation}
    f(x) \in \left\{z : \begin{bmatrix}
        \mathcal{C}_{\mathcal{O}_{i,0}}\\
        .\\
        .\\
        .\\
        \mathcal{C}_{\mathcal{O}_{i,n}}
    \end{bmatrix}^T z > 0 \text{, for all }i: \text{row}_i(\mathcal{C}_{\mathcal{O}})
     \cdot x = \Vec{d}_{\mathcal{O}_i} \right\}
\end{equation}
\end{corollary}

\begin{theorem}
Consider a linear time-invariant system of the form $\Vec{\dot x} = Ax + \Vec{b}$.
Let's define $\mathcal{O} = \left\{x : c^T x \leq d \right\}$ and 
$x_{\lambda} = \lambda \cdot x_1 + (1-\lambda) \cdot x_2$ where $\lambda$
is a scalar.

Then,  
\begin{equation}
    \left \{
    \begin{array}{ll}
        c^T \cdot f(x_1) > 0\\
        c^T \cdot f(x_2) > 0
    \end{array}
    \right. \Rightarrow c^T \cdot f(x_\lambda) > 0
\end{equation}

\end{theorem}